% beamerstructure2 presentation

\documentclass{beamer}
\usetheme{Warsaw}

\setbeamertemplate{navigation symbols}{}
\title{Sectionally complemented chopped lattices}
\author{George Gr\"atzer\inst{1}
\and Harry Lakser\inst{1}
\and Michael Roddy\inst{2}}
\institute{\inst{1} University of Manitoba
\and \inst{2} Brandon University}
\date{Conference on Lattice Theory, 2006}
%Note that multiple authors are separated by \verb+\and+ and so are
%the various institutions. The \verb+\date+ command is treated as a 
%footnote.

\begin{document}

\begin{frame}
\titlepage
\end{frame}

%To show an outline of the whole presentation, we have to fake it. 
%The command \verb+\tableofcontents+ provides you with the Table of
%Contents of the whole presentation only if there are no parts. 
%With an optional argument, it can also provide the Table of Contents
%of a specific part: \verb+\tableofcontents[part=3]+ is the 
%Table of Contents of Part 3. We also use the \ttt{pausesections} 
%option.

%The Table of Contents of the whole presentation is given in an 
%unnumbered section and three unnumbered subsections.

\section*{Outline}

\subsection*{Part I: Background}

\begin{frame}
\frametitle{Outline of Part I: Background}

\tableofcontents [part=1,pausesections]
\end{frame}

\subsection*{Part II: Characterizing the 1960 sectional complement}

\begin{frame}
\frametitle{Outline of Part II: Characterizing the\\1960 sectional complement}

\tableofcontents[part=2,pausesections]
\end{frame}

\subsection*{Part III: The general problem}

\begin{frame}
\frametitle{Outline of Part III: The general problem}

\tableofcontents[part=3,pausesections]
\end{frame}

%Now come the three parts. Each part is introduced with two frames: 
%the first with \verb+\partpage+ (which is the ``titlepage'' for the
%part) and the second with the command \verb+\tableofcontents+ 
%(for the part).

%We provide each section with a frame to activate it.

\part{Background}

\begin{frame}
\partpage
\end{frame}

\begin{frame}
\frametitle{Part I\\Outline}

\tableofcontents
\end{frame}

\section{Chopped lattices}

\begin{frame}
\frametitle{Defining chopped lattices}

Starting the definitions
\end{frame}

\section{Ideals and congruences}

\begin{frame}
\frametitle{Ideals}

Continuing the definitions
\end{frame}

\part{Characterizing the 1960 sectional complement}

\begin{frame}
\partpage
\end{frame}

\begin{frame}
\frametitle{Part II\\Outline}

\tableofcontents
\end{frame}

\section{What it is not}

\begin{frame}
\frametitle{Not maximal, minimal, or fixed point}

Counterexamples
\end{frame}

\section{The characterization theorem}

\begin{frame}
\frametitle{The main result}

State the characterization theorem
\end{frame}

\part{The general problem}

\begin{frame}
\partpage
\end{frame}

\begin{frame}
\frametitle{Part III\\Outline}

\tableofcontents
\end{frame}

\section{The Lakser Theorem}

\begin{frame}
\frametitle{The problem}

Stating the general problem and Harry's observation
\end{frame}

\section{A small counterexample}

\begin{frame}
\frametitle{Four-element overlap}

Counterexample
\end{frame}

\section{A cyclic counterexample}

\begin{frame}
\frametitle{Three cycle}

Cyclic counterexample
\end{frame}
\end{document}