%First document, firstdocumentmod.tex
\documentclass{amsart}
\usepackage{amssymb,latexsym}

\newcommand{\pdelta}{\pmod{\delta}}
\DeclareMathOperator{\length}{length}
\newtheorem{lemma}{Lemma}

\begin{document}
\title{A technical lemma\\ for congruences of finite lattices}  
\author{G. Gr\"{a}tzer} 
\address{Department of Mathematics\\
  University of Manitoba\\
  Winnipeg, MB R3T 2N2\\
  Canada}
\email[G. Gr\"atzer]{gratzer@me.com}
\urladdr[G. Gr\"atzer]{http://tinyurl.com/lej5g49}
\date{March 21, 2014}
\subjclass[2010]{Primary: 06B10.}
\keywords{finite lattice, congruence.}
\begin{abstract}
We present here a Technical Lemma 
for congruences on \emph{finite lattices}.
This is not difficult to prove but it has already has proved 
its usefulness in some applications.
\end{abstract}
\maketitle

\subsection*{Introduction}\label{Intro}%Section~\label{Intro}
In some recent research, G. Cz\'edli
and I, see \cite{gC13} and \cite{GS13}, spent quite an effort
in proving that some equivalence relations 
on a planar semimodular lattice
with intervals as equivalence classes are congruences. 
The number of cases we had to consider
was dramatically cut by the following result.

\begin{lemma}\label{L:technical}%Lemma~\ref{L:technical}
Let $L$ be a finite lattice. 
Let $\delta$ be an equivalence relation on $L$
with intervals as equivalence classes.
Then $\delta$ is a congruence relation if{}f 
the following condition and its dual hold:
\begin{equation}\label{E:cover}%\eqref{E:cover}
\text{If $x$ is covered by $y,z \in L$ 
and $x \equiv y \pdelta$,
then $z \equiv y + z \pdelta$.}\tag{C${}_{+}$}
\end{equation}
\end{lemma}

\subsection*{Proof}\label{Proof}%Section~\label{Proof}
We prove the join-substitution property:  
if $x \leq y$ and $x \equiv y \pdelta$, then
\begin{equation}\label{E:Cjoin}%\eqref{E:Cjoin}
x + z \equiv y + z \pdelta.
\end{equation}
Let $U = [x, y+ z]$.
We induct on $\length U$, the length of $U$.  

Let $I=[y_1,y+ z]$ and $J=[z_1,y+ z]$. 
Then $\length I$ and $\length J  < \length U$. 
Hence, the induction hypothesis applies to $I$ 
and $\delta\rceil I$, and we obtain that 
$w \equiv y+ w \pdelta$. 
By the transitivity of $\delta$, we conclude that 
\begin{equation}\label{E:three}%\eqref{E:three}
z_1 \equiv y+ w \pdelta.
\end{equation}
Therefore, applying the induction hypothesis to $J$ 
and $\delta \rceil J$, we conclude from \eqref{E:three} that 
\[
   x+ z = z + z_1 \equiv z + (y+ w) = y+ z \pdelta,
\] 
proving \eqref{E:Cjoin}.

\begin{thebibliography}{9}
\bibitem{gC13}%G. Cz\'edli~\cite{gC13}
G. Cz\'edli,
\emph{Patch extensions and trajectory colorings of slim
rectangular lattices.}
Algebra Universalis, to appear. 

\bibitem{GS13}%G. Gr\"atzer \cite{GS13}
G. Gr\"atzer, 
\emph{Congruences of fork extensions of lattices.}
Acta Sci. Math. (Szeged), submitted. arXiv: 1307.8404
\end{thebibliography}
\end{document}